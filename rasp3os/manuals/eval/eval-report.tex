\documentclass[10pt]{article}
%\usepackage{iwpt05}
\usepackage{harvard}
\usepackage{times}
\usepackage{latexsym}
\usepackage{graphics}
\usepackage{epsfig}
\usepackage{rotating}
\usepackage{lscape}
\usepackage{url}
\usepackage{psfrag}
\usepackage{arydshln}
\usepackage{subfigure}

\bibliographystyle{agsm}

\newcommand{\stt}[1]{{\small\texttt{#1}}}
\newcommand {\etal} {{\it et al.}}

\newtheorem{theorem}{Theorem}[section]
\newtheorem{lemma}[theorem]{Lemma}
\newtheorem{proposition}[theorem]{Proposition}
\newtheorem{corollary}[theorem]{Corollary}

\newenvironment{proof}[1][Proof]{\begin{trivlist}
\item[\hskip \labelsep {\bfseries #1}]}{\end{trivlist}}
\newenvironment{definition}[1][Definition]{\begin{trivlist}
\item[\hskip \labelsep {\bfseries #1}]}{\end{trivlist}}
\newenvironment{example}[1][Example]{\begin{trivlist}
\item[\hskip \labelsep {\bfseries #1}]}{\end{trivlist}}
\newenvironment{remark}[1][Remark]{\begin{trivlist}
\item[\hskip \labelsep {\bfseries #1}]}{\end{trivlist}}

\newcommand{\qed}{\nobreak \ifvmode \relax \else
      \ifdim\lastskip<1.5em \hskip-\lastskip
      \hskip1.5em plus0em minus0.5em \fi \nobreak
      \vrule height0.75em width0.5em depth0.25em\fi}

\title{RASP Evaluation Schemes}
\author{Rebecca Watson\\
Natural Language and Information Processing Group\\
University of Cambridge\\
Computer Laboratory\\
Rebecca.Watson@cl.cam.ac.uk}
\date{\today}
 
\begin{document}

\hyphenation{machine-readable self-contained}
\maketitle
\begin{abstract}
This report outlines details of the relational
dependency evaluation schemes written for RASP grammatical
relations.
\end{abstract}

\setcounter{tocdepth}{2}

\tableofcontents

%\listoftables
%\listoffigures

\newpage

\section{Introduction}

The Robust Accurate Statistical Parsing (RASP) system 
for English developed by \citename{briscoe02}~\citeyear{briscoe02} is 
publicly available for academic use. 
The parser is capable of returning
a number of different output formats including grammatical relations
(GRs). For details of the manually written tag sequence grammar 
and details of the GRs readers are referred to~\cite{briscoe05a}. 

This report contains details of the relational
dependency based evaluation schemes enabling the reader to either
a) replicate the schemes or 
b) interpret results output by the evaluation system.
The test and gold standard files employed %(Section~\ref{gold})
and the evaluation system outlined herein have been made publicly available 
for research to enable comparison of research efforts.

\citename{briscoe05a}~\citeyear{briscoe05a} outlines details of
the GRs output by RASP. 
All GRs take the following form:
\begin{verbatim}
(GR-type subtype head dependent initial-GR)
\end{verbatim}
The first item in this list: the \stt{GR-type}, specifies 
the type of relationship between the \stt{head}
and \stt{dependent} 
(e.g. (\stt{subj})ect, (\stt{obj})ect, (\stt{mod})ifier). 
The remaining items in the list:
(\stt{subtype, head, dependent, initial-GR}) will be
referred to as `slots' herein. 
The \stt{subtype} and \stt{initial-GR} are specified for only a subset of
all possible GR-types. That is, their presence depends on the GR-type. 
In the case that these optional 
slots are present for a given GR-type the slot may appear with 
the default (unspecified) value of ``\_''. 

For all evaluation schemes, both the GR-type and slot values 
(the value in the specified slot) must be `equal'.
The set of evaluation schemes differ only in their definition
of equality between GR-types and the set of slots to compare
for slot `equality'. That is, they differ in the criteria 
for whether the test relation is correct: 
the test and standard items `match'. 
Section~\ref{eval} outlines
the overall architecture of the evaluation schemes while
Section~\ref{grmatch} and Section~\ref{slot-compare}
outline alternative definitions of GR-type matching and the
set of slots to compare, respectively.

We need to consider alternative sets of slots to compare as
other parsers specify only head and dependent slots. Therefore, the
evaluation system should be capable of comparing these slots only, 
to enable us to compare system performance.

There are 17 GR-types that form an inheritance 
hierarchy enabling relations to be underspecified.
Readers are referred to \cite{briscoe05a} for
a figure of the hierarchy.
This hierarchy of relations %is shown in \cite{briscoe05a}.
is shown in Figure~\ref{gr-hier}. Note that the \stt{passive} GR-type
is not included in evaluation as this GR-type represents
information available in the \stt{ncsubj} GR-type when the
\stt{initial-GR} value is \stt{obj}.
This hierarchy enables a number of different schemes to be considered
as we can define alternative definitions of GR-type match using
this hierarchy. 
For example, we may consider GR-types in the test set (test, henceforth) 
that subsume GR-types in the gold standard set (standard, henceforth) 
as correct.

\begin{figure}
\centering
{\tt    \setlength{\unitlength}{0.65pt}
{\small \begin{picture}(520,230)(-10,5)
\put(-10,5){\framebox(520,230){}}
\thinlines

              \put(354,85){\line(-3,-1){62}}
              \put(354,85){\line(0,-1){22}}
              \put(354,85){\line(3,-1){69}}

              \put(436,50){\line(3,-2){36}}
              \put(436,50){\line(-3,-2){36}}

              \put(289,50){\line(2,-1){47}}
              \put(289,50){\line(0,-1){24}}
              \put(289,50){\line(-5,-2){53}}

              \put(180,72){\line(2,-1){40}}
              \put(180,72){\line(0,-1){18}}
              \put(180,72){\line(-2,-1){40}}

              \put(310,135){\line(3,-2){56}}
              \put(310,135){\line(-1,-2){10}}
              \put(310,135){\line(-5,-2){116}}

              \put(308,98){\line(-1,-1){67}}
              \put(308,98){\line(-6,-1){107}}

              \put(159,168){\line(-3,-2){37}} % argmod
              \put(159,168){\line(6,-1){134}}

              \put(95,128){\line(3,-1){68}} % mod
              \put(95,128){\line(3,-4){17}}
              \put(95,128){\line(-3,-4){17}}
              \put(95,128){\line(-3,-1){68}}

              \put(206,210){\line(-5,-1){124}} % dependent
              \put(206,210){\line(-3,-2){37}}
              \put(206,210){\line(5,-2){62}}
              \put(206,210){\line(4,-1){100}}
              \put(206,210){\line(6,-1){148}}

              \put(176,218){dependent} % \small\it 

              \put(70,174){ta}
              \put(147,174){arg\_mod}
              \put(258,174){det} % \small\it 
              \put(298,174){aux} % \small\it 
              \put(343,174){conj} % \small\it 

              \put(95,134){mod}
              \put(300,141){arg} 

              \put(0,94){ncmod}
              \put(49,94){xmod}
              \put(90,94){cmod}
              \put(134,94){pmod}

              \put(260,104){subj\_dobj} % \small\it 

              \put(170,80){subj}
              \put(345,90){comp} % \small\it 

              \put(101,44){ncsubj}
              \put(160,44){xsubj}
              \put(208,44){csubj}

              \put(280,54){obj} % \small\it 
              \put(340,54){pcomp}
              \put(411,54){clausal}

              \put(220,14){dobj}
              \put(277,14){obj2}
              \put(322,14){iobj}

              \put(385,14){xcomp}
              \put(455,14){ccomp}
              
\end{picture}}}
\caption{The GR hierarchy}
\label{gr-hier}
\end{figure}

%Section~\ref{original} outlines the  previous (`original') evaluation
%scheme. Full details of the evaluation scheme are provided and the
%short-comings of this scheme are addressed in Section~\ref{new} in
%which a number of alternative schemes are outlined to address
%these short-comings.

Section~\ref{perf} reports current system performance using
each of the evaluation schemes. Finally, Section~\ref{resources}
provides details of the files that are publicly available including
the gold standard files and the file formats required to utilise
the evaluation script.

\section{Evaluation System}
\label{eval}

This section outlines the evaluation system's architecture
and highlights the functions which differ between each evaluation
scheme (scheme, henceforth).
These functions refer to the whether the GR-types match
and which slots to compare
(the definition of a slot match is constant between schemes).
Once the reader is familiar with the general architecture,
we will discuss the alternative sets of slots that may be compared
and the GR-Type match options
in Sections~\ref{slot-compare} and~\ref{grmatch}, respectively. 
Finally, Section~\ref{resources} will outline the script file(s)
that can be called to invoke the evaluation system over a set of
test files and the script parameters that
will select the scheme utilised by the system.

Each sentence is evaluated in-turn whereby the test and standard
GR sets are compared. The test and standard files are required
to contain the same number of sentences. An error message
will be output by the system if this is not the case. 
The format required for test files is outlined in Section~\ref{resources}.

A GR in the test set `matches' a GR in the standard set if the test
GR is considered to be correct in the given scheme.
Duplicate GRs are permitted in the standard set but not in the test set. 
If duplicate GRs occur in the test set (and not in the standard)
then these are considered incorrect as only one match 
is permitted against each GR in the standard.

The evaluation system correlates for each sentence:
\begin{itemize}
\item {\bf common}: the set of GRs that matched between the test and
  standard, where each element in this list are of the form: 
\{standard GR, test GR\}.
\item {\bf missing}: the set of GRs in the standard for which no test
GRs were found to match.
\item {\bf extra}: the set of GRs in the test set for which no
  matches were found in the standard set.
\end{itemize}
After processing each sentence, the evaluation system outputs
the list of common, missing, and extra GRs to the output file.

The evaluation system correlates across all sentences in the test
set\footnote{Note that the std-total- tst-total- and agree- scores
are recorded for each GR-type}:
\begin{itemize}
\item {\bf nsents}: the number of sentences in the test set.
\item {\bf std-total-\emph{GR-type}}: 
the number of standard GRs of type \emph{GR-type}.
\item {\bf tst-total-\emph{GR-type}}: 
the number of test GRs of type \emph{GR-type}.
\item {\bf agree-\emph{GR-type}}: 
the number of GRs of type \emph{GR-type} for which a match was found.
\end{itemize}

\subsection{Output Formats}

This section briefly outlines the alternative output formats
available. The output file specified (see Section~\ref{resources})
contains for each sentence: the sentence number, the sentence, 
the list of common GRs (``In both''), the missing GRs (``Standard
only''), and the extra GRs (``Test only'') outlined in the previous
section. 
The sentence summary output format outlined in Section~\ref{sentsum} 
is then output to the file. 
The micro- and macro-averaged scores for each relation (and
for the test set overall as outlined in Section~\ref{sum}) 
are then output to the file (outlined in Section~\ref{relsum}).

\subsubsection{Relation Summary}
\label{relsum}

The relation summary format reports the precision, recall, and F$_{1}$ for 
each GR-type. For all but one option these are computed by percolating
up the counts for std-total- tst-total- and agree- up the heirarchy,
reflecting the average performance across each GR-type and it's
associated sub-types in the hierarchy. However, for the hierarchy method
outlined in Section~\ref{hier}, these counts are not percolated
upwards in a routine fashion. Section~\ref{hier} outlines full details
of how the counts are percolated upwards.

Once the counts have been percolated upwards in the hierarchy
we can determine precision, recall, and F$_{1}$ for each GR-type
using:
\begin{equation}
precision = \frac{agree-GR-type}{tst-total-GR-type}
\end{equation}
\begin{equation}
recall = \frac{agree-GR-type}{std-total-GR-type}
\end{equation}
\begin{equation}
F_{1} =  \frac{2 \times precision \times recall} {precision + recall}
\end{equation}

Note that the relation summary includes the summary output outlined
in Section~\ref{sum}.

\subsubsection{Summary}
\label{sum}

The summary format reports the micro-averaged and macro-averaged
precision, recall, and F$_{1}$ scores over all relations. 
The micro-averaged scores represent the performance of these 
performance measures considered over all GR-types. Thus, these
scores are identical to the scores reported for the
\stt{dependency} relation in the relation summary format outlined
in the previous section.

The macro-averaged scores represent the average of each score
over the 17 GR-types. 
However, these scores are not calculated using
the percolated counts. 
Precision, recall and F$_{1}$ are determined for each GR-type 
using the raw counts of std-total- tst-total- and agree- 
over each GR-type. The average of each of these measures over
each GR-type is then calculated to report the `macro' 
average of the scores.
These scores place equal importance over each GR-type rather
than over those that occur more frequently.

\subsubsection{Sentence Summary}
\label{sentsum}

This format outputs precision, recall and F$_{1}$
for each sentence in the test set. These scores are determined
for each sentence using the equations outlined in
Section~\ref{relsum}. 

\subsection{GR Match}

This section outlines the definition of a `match' between
two GRs (test and standard). This definition holds over all
GR-type definitions except for the hierarchy method
outlined in Section~\ref{hier}, which describes
slight modifications to this definition.

In order for the test GR (tst) to match a given standard GR (std)
the following must hold:
\begin{itemize}
\item the GR-types of std and tst match and
\item the slots (the values in the slots) match.
\end{itemize}
As previously mentioned, the definition of a GR-type match
differs between schemes and the set of slots to compare differs.
Section~\ref{slots} will define the slots that occur in each GR-type
and Section~\ref{slotequality} gives the definition of 
a match between two given slots.

\subsubsection{Slots}
\label{slots}

Figure~\ref{gr-slots} illustrates the list of 16 possible GR-types
(we do not include the \stt{passive} GR-type in evaluation)
and the set of associated slots. For each GR-type all and only these
slots must occur. Though the slot values may 
contain ``\_'' if the optional slot is unspecified. The system
will output an error message if a GR-type does not occur with
the specified number of slots shown in the figure.

\begin{figure*}
{\small
\begin{center}
\begin{verbatim}
(dependent subtype head dependent)
(mod       subtype head dependent)
(ncmod     subtype head dependent)
(xmod      subtype head dependent)
(cmod      subtype head dependent)
(pmod              head dependent)
(det               head dependent)
(arg_mod   subtype head dependent)
(arg       subtype head dependent)
(subj              head dependent initial-gr)
(ncsubj            head dependent initial-gr)
(xsubj             head dependent initial-gr)
(csubj             head dependent initial-gr)
(subj_dobj         head dependent)
(comp              head dependent)
(obj               head dependent)
(dobj              head dependent)
(obj2              head dependent)
(iobj              head dependent)
(clausal           head dependent)
(xcomp     subtype head dependent)
(ccomp     subtype head dependent)
(pcomp             head dependent)
(aux               head dependent)
(conj              head dependent)
(ta        subtype head dependent)
%(passive        head)
\end{verbatim}
\end{center}}
\caption{The GR-types with associated required slots. }
\label{gr-slots}
\end{figure*}

\subsubsection{Slot Match}
\label{slotequality}

A number of options are available for the evaluation
system. One option outlines which set of slots to compare. However,
the definition of a slot match remains constant across 
these options.
Two slots (of the same type) are considered to match if
one of the following constraints hold given the slot values
tst-arg and std-arg for test and standard slots, respectively:
\begin{enumerate}
\item the values tst-arg and std-arg are identical (including
unspecified value of \_).
\item tst-arg and std-arg are both specified (not \_) and:
\begin{enumerate} 
\item either tst-arg or std-arg specifies the value
 \stt{ellip}, signifying that that the argument references an ellipsis or
\item{\label{NE-match} std-arg is a word in the multiword tst-arg. 
A multiword represents an NE in the text where words are separated
by \_.}
\end{enumerate} 
\item either tst-arg or std-arg are unspecified (\_) and:
\begin{enumerate}
\item the slot is of type \stt{type} and
\item the GR-type is one of 
\stt{mod, ncmod, xmod, cmod, pmod, arg, xcomp, ccomp, ta}.
\end{enumerate}
\end{enumerate}

Note that the slot match criteria~\ref{NE-match}
accounts for the use of named-entity recognition 
during parsing. That is, the test files
include versions of the text and gold standard using
NE markup. In this case the last word in the NE item appears
in the gold standard. When parsing RASP will output
these multiword constituents as a single word
where words are separated by ``\_''.
For example, the second sentence in the test file 
is\footnote{The beginning of sentence marker 
is $\wedge$.}:
\begin{verbatim}
^ The following issues were recently filed with the 
<w>Securities and Exchange Commission</w>:
\end{verbatim}

In this case, the test file (the .parses file outlined
in Section~\ref{resources}) should include GRs containing
the multi-word item `Securities\_and\_Exchange\_Commission' 
to enable the slot value to match with the standard 
GRs for the sentence:
\begin{verbatim}
(ncsubj filed issues obj)
(ncmod _ filed recently)
(aux filed were)
(passive filed)
(iobj filed with)
(dobj with Commission)
(det Commission the)
(det issues The)
(ncmod _ issues following)
\end{verbatim}

\section{Slot Comparison}
\label{slot-compare}

The definition of a slot match was given in Section~\ref{slotequality}.
In this section we will outline the alternative slot comparison
options available. That is, options that define which set of slots
to compare. All slots in the set compared must match for 
the slots to match. 

These options, outlined in the following sections, 
can be summarised as follows:
\begin{itemize}
\item All: all slots are compared.
\item Head-Dependent: only head and dependent slots are compared.
\item Head-Dependent-NCSUBJ: in addition to comparing the head and dependent slots,
we compare the \stt{initial-GR} slot for the \stt{ncsubj}
GR-type if the value of the slot is \stt{obj} in either
the test or standard.
\end{itemize}

\subsection{All}

For the ALL slot comparison option, 
all slots are compared between the standard and test
relation.
For each slot in the standard relation, we determine
whether or not this slot is present in the test relation.
If the slots are both present in the test relation then we
compare the slots.
If a slot is not present in either the test or standard
relation then the slot is considered correct by
default.
All slots that are present in both the standard and test
relation are required to match.

For example, if the standard relation and the
test relation specify \stt{ncmod} and \stt{iobj}, respectively,
then we will compare the slots \stt{head} and \stt{dependent}
only and we will ignore the \stt{type} slot. If the values in the
\stt{head} and \stt{dependent} slots match
then the slots are considered to match.

Note that the original evaluation scheme uses a modified version of this
definition: the number of slots in both the standard and test relations
must be equal. However, the scheme does not specify that these slots must
contain the same set of slot types. We relaxed this original definition for the
new schemes (as outlined above) to enable comparison between 
the slots of any two GR-types.

\subsection{Head-Dependent}

For the HEAD-DEPENDENT option, only the head and dependent slots
are compared.

\subsection{Head-Dependent-NCSUBJ}

For the HEAD-DEPENDENT-NCSUBJ option, the \stt{head} and 
\stt{dependent} slots are compared. 
Further, if the GR-type is \stt{ncsubj} then
we compare the \stt{initial-GR} slot if the slot value 
is \stt{obj} in either the test or standard.

\section{GR-Type Match}
\label{grmatch}

This section outlines the various options for determining
a GR-type match. That is, whether or not the GR-type of
the standard relation (std-GR) and test relation (tst-GR) match. 
These options, outlined in the following sections, 
can be summarised as follows:
\begin{itemize}
\item Original: the tst-GR can be
\begin{itemize} 
\item equal to the std-GR,
\item more general than the std-GR by one level in the hierarchy or
\item more specific than the std-GR by any number of levels in the hierarchy.
\end{itemize}
\item Equality: the tst-GR must be equal to the std-GR.
\item Subsumption: the tst-GR can be:
\begin{itemize} 
\item equal to or 
\item more general than the std-GR by any number of levels in the
  hierarchy.
\end{itemize}
\item Hierarchy: this scheme returns the most specific GR-type in the hierarchy
that subsumes both the tst-GR and std-GR. A match is always
returned and, thus, a match is considered to occur at this level in the
hierachy.
\end{itemize}

%Precision, recall, and F$_{1}$ is calculated for each GR-type in the
%hierarchy. Counts are percolated upwards so that the performance
%of the GR-type \stt{dependency} is in effect the average
%performance of each GR-type for which test and/or standard GRs
%occured.

\subsection{Original}

The original GR-type match can be summarised as follows, the tst-GR can be:
\begin{itemize} 
\item equal to the std-GR,
\item more general than the std-GR by one level in the hierarchy or
\item more specific than the std-GR by any number of levels in the hierarchy.
\end{itemize}

The original scheme enables the tst-GR to be more general by one level
in the hierarchy, and only if the std-GR is a leaf node in the hierachy.
Note that the current standard does not contain
any GR-types that are not leaf nodes in the hierachy. Therefore,
the test GR-type will never be more specific than the standard.

\subsection{Equality Based}

The original GR-type match scheme outlined in the previous section
enables the GR-type to be more general than the standard. Therefore,
the reported performance does not provide information on how
well the parser is determining a particular GR-type.
For example, we may wish to know how well the system performs 
in terms of the \stt{ncmod} GR-type.
Currently, if the standard contains a \stt{ncmod}, the test
may return a \stt{mod} or an \stt{ncmod}. Thus the user 
does not know the individual counts of these two scenarios. Two
systems will seem to perform equally as well even if one returns
the more precise match \stt{ncmod} more frequently than
 the other.
Therefore, the equality based GR-type match measures the exact performance 
of each GR-type. That is, it performs GR-type matching by requiring the GR-type 
to be equal.

\subsection{Subsumption Based}

As previously mentioned, 
the GR-types present in the standard are only those that occur 
as leaf nodes in the hierarchy. That is, the standard contains
only the most specific GR-type possible. 
Thus, the test relation will never be more specific 
than the standard and it would be more beneficial 
to determine performance by enabling test relations 
to be more general by \emph{any degree} in the hierachy.
Therefore, the subsumption based GR-type match considers
the tst-GR and std-GR to match if
tst-GR is equal to or more general than the std-GR.

\subsection{Hierarchy Based}
\label{hier}

While the previous schemes enable the tst-GR to
be either equal to and optionally more general than the std-GR,
they do not report the unlabelled dependency accuracy.
The unlabelled dependency performance reflects how well we 
determine any type of dependency relation. That is,
how well we perform just on matching the required slots.
Many parsers report this measure and it would be beneficial for
comparison if the evaluation system reported this performance.

The unlabelled dependency performance reflects how well we
do at the level of \stt{dependent} in the hierarchy. That is,
if we enable any two GR-types to match and define a GR match
to occur if the required slots all match. 
Similarly, we may wish to determine how well we do 
at the level of \stt{mod}.
At \stt{mod} level of the hierarchy we can define that
two GR-types match if they are both one of: \stt{mod, ncmod,
cmod, pmod}. Determining a score for \stt{mod} in this way
enables us to compare the performance of RASP at determining
modifiers to another parser that does not make the fine-grained
distinction between modifier types. That is, we do not penalise
RASP for trying to make finer-grained distinctions.

Thus, at any level in the hierarchy we wish
to see how well we do if we enable GR-type match to occur between
any two GR-types at that level or lower in the hierarchy.
We determine the `GR-type match' as the 
the most specific GR-type in the hierarchy
that subsumes both the tst-GR and std-GR GR-types. 
A GR-type match is always returned for a given test GR (tst-GR) 
and list of standard GRs (std-GR-list). 
As previously mentioned, a GR match is determined between a test
and standard GR if the GR-types match and the set of required
slots match.
However, we do not wish to match to the first item in the std-GR-list
for which the slots match. Instead, we wish to match the tst-GR
to a standard GR for which the slots match and also the GR match 
(the GR-type that subsumes both the test
and standard GR) is the most specific.

Given that the relation tst-GR finds a matching std-GR 
(the slots match) with the GR-type match \emph{grtype-common},
we percolate up counts for std-total- tst-total- and agree-
GR-types as follows:
\begin{itemize}
\item std-total-\emph{GR-type}: increment this count for 
the given GR-type in the hierarchy if the \emph{GR-type}
is equal to or subsumes the std-GR GR-type. 
\item tst-total-\emph{GR-type}: increment this count for 
the given GR-type in the hierarchy if the \emph{GR-type}
is equal to or subsumes the tst-GR GR-type.
\item agree-\emph{GR-type}: increment this count for 
the given GR-type in the hierarchy if the \emph{GR-type}
is equal to or subsumes the grtype-common.
\end{itemize}

Note that for each GR in `missing' or `extra' lists outlined
in Section~\ref{eval} we also percolate up the counts for
std-total- and tst-total-, respectively, 
for each GR-type as above.


\section{Resources}
\label{resources}

This section outlines all of the key files which
comprise or are required by the evaluation system.
Section~\ref{gold} gives details of the gold standard files
and required format of the test files.
These resources can be found in \$RASP/extra/neweval 
(\$RASP-EVAL, henceforth).
Example files have been included in the \$RASP-EVAL
directory: example.out, example.grtext, example.parses
and example.output as well as the NE versions of each.

\subsection{Test Files and Gold Standard}
\label{gold}

\subsubsection{Gold Standard - Parc700 Depbank}

\citename{king03}~\citeyear{king03} outline the development of the PARC 700 Dependency
Bank (henceforth, DepBank), a gold-standard set of relational
dependencies for 700 sentences (originally from the Wall Street 
Journal) drawn at random from Section 23 of the Penn Treebank. 
\citename{briscoe05}~\citeyear{briscoe05} extend DepBank with a set of
gold-standard GRs and (manually corrected) PoS tags.
The extended DepBank file can be found at
\$RASP-EVAL/gold700files.rasp, note that this file includes
all 700 sentences.

We will use the gold-standard GRs to measure parser accuracy,
over the same 560 sentence test suite from the DepBank 
utilised by \citename{kaplan04}~\citeyear{kaplan04}.
The 560 test suite subset and the NE version of this subset 
can be found in
\$RASP-EVAL/parc700/gold560.rasp and in 
\$RASP-EVAL/parc700/gold560ne.rasp, respectively.

\subsubsection{Text and Pre-processed Files}

The text file (to parse) and the NE version of this text file 
can be found in \$RASP-EVAL/parc700/test.not-ne and in 
\$RASP-EVAL/parc700/test.ne, respectively.
These files have sentence boundaries automatically detected
and the -s option should be used when invoking \$RASP/scripts/rasp.sh
over these files.

Alternatively, the user can employ  \$RASP/scripts/rasp\_parse.sh
using the pre-processed versions of test.not-ne and test.ne:
\$RASP-EVAL/parc700/test.not-ne.stag.data and 
\$RASP-EVAL/parc700/test.ne.stag.data, respectively. 
Note that these files contain the pre-processed text using the PoS
tagger in forced-choice mode.

\subsection{Running the Evaluation System}

Two scripts are provided: eval.sh and eval\_system.sh. 
This section outlines the details of utilising these
script files that invoke in the evaluation system.
The eval.sh script is designed to pre-process the 
RASP output file ready for the eval\_system.sh script.
The resulting file formats prepared by eval.sh are
outlined in Section~\ref{inputformats}. 
The eval\_system.sh script accepts these pre-processed
input files and a number of input parameters that
define the evaluation options including the set
of slots to compare and the definition of GR-type
match as outlined previously.

The input parameters for each script are described in
Section~\ref{scriptparam}. 
Example invocation (the same evaluation scheme is applied in each) 
over parser out file example.out:
\begin{verbatim}
./eval.sh -d -t example -e'-u -o -s n -g s -m'
./eval_system.sh -t example -c sents-tst.gr -u -o -s n -g s -m
\end{verbatim}

\subsubsection{Input Formats}
\label{inputformats}

\$RASP-EVAL/eval.sh takes the RASP output format in file x.out (see e.g.
\$RASP-EVAL/example.out) and produces x.trans.grtext and
x.trans.parses (or x.grtext and x.parses depending on which
evaluation options are utilised)
in the format required by \$RASP-EVAL/eval\_system.sh. These files
contain the sentences and GR sets from x.out, respectively.

This section outlines the required formats for .grtext and
.parses files to enable users to utilise the evaluation
system with other parsers, if required. The next section outlines the
input parameters available to each script.

Users who employ the RASP parser can utilise 
the eval.sh script to automatically
create these input files from RASP output file x.out.
Eval.sh then invokes the eval\_system.sh script with these
input files.

Note that the files input into eval\_system.sh must contain
GRs in which the original word occurs and not the morphological
variant of the word e.g \$RASP-EVAL/parc700/test.not-ne.stag
compared to \$RASP-EVAL/parc700/test.not-ne.stag.data. 
The eval.sh script automatically converts the GR words 
from e.g. asset+s to assets using the eval\_transform.sh script. 
If the user wishes to parse
the \$RASP-EVAL/parc700/test.(not-)ne.stag file instead which
does not include morphological variants (so that RASP will
output GRs with the original word forms) then the user should
use the -x option for eval.sh outlined below. However, performance
using subcategorisation or phrasal verbs (using the rasp\_parse.sh
options -s and -x respectively) will not correspond to the true
performance as the stem of the verb is required by RASP in these
cases. Therefore, the user is directed to parse the 
\$RASP-EVAL/parc700/test.(not-)ne.stag.data files which include
the morphological variants of words.

\begin{itemize}
\item{x.grtext}

An example of the format required is shown in
\$RASP-EVAL/example.grtext. The file is required to have the sentence
number (starting at 1) followed on the next line by the sentence
itself. The sentence should be tokenised into the set 
that specifies the possible items that may appear in the GR slots
as values. The evaluation system will output an error if a value
occurs in a GR slot that is not present in this sentence.
A blank line should follow each sentence.

\item{x.parses}

An example of the format required is shown in
\$RASP-EVAL/example.parses. The example file begins with two lines
that can be ignored and these lines do not necessarily have to
occur in the .parses file. All input prior to the first sentence
(the token 1 shown on line 4 of example.parses) will be ignored.
Each sentence should be numbered (starting from 1) followed by
a blank line and then the GR set for the sentence (where one GR
occurs per line).
A blank line should follow each GR set.
\end{itemize}

Both the x.parses and x.grtext contain information for each sentence,
where the information is labelled by the sentence number.
Note that each file input to eval\_system.sh should specify the
same number of sentences otherwise an error will be returned by the
system. Further, the evaluation system ensures that the same sentence
number is compared from x.grtext, x.parses and the standard file.
Therefore, these files must contain the sentence information labelled
with the correct sentence number (sentence numbering starts at 1 for
all files).

\subsubsection{Script Parameters}
\label{scriptparam}

Both eval.sh and eval\_system.sh can be used to invoke the
evaluation system (in eval\_system.sh). The eval.sh script
should be used if the user first wishes to create the input
files x.grtext and x.parses from the RASP output file x.out.
The switches all have a default value of false.
Example invocations (calling equivalent schemes 
in the evaluation system):
\begin{verbatim}
./eval.sh -d -t example -e'-u -o -s n -g s -m'
./eval_system.sh -t example -c sents-tst.gr -u -o -s n -g s -m
\end{verbatim}

{\bf Eval.sh parameters:}
\begin{itemize}
\item {\bf -t $<$test-file$>$}

This option is used to specify the the test file name (test-file). 
That is, the x in the RASP output file x.out. 
Note that if the -r switch is used the test file 
specified will be created by parsing the appropriate test suite.

\item {\bf -r : run rasp switch}

This switch specifies to run rasp over the text file
\$RASP-EVAL/parc700/test.not-ne 
to create the specified test-file.out file.

\item {\bf -n : NE switch}

This switch specifies that sentences with NE mark-up 
were or are to be parsed.
Therefore, the file sent-ne-tst.gr will
be used as the standard in place of the sents-tst.gr.
Note that these standard files are automatically created
by eval.sh using either \$RASP-EVAL/parc700/gold560.rasp or
\$RASP-EVAL/parc700/gold560ne.rasp. Therefore,
the most recent version of the gold standard(s) will
be employed during evaluation.

Used in conjunction with the -r switch results
in RASP parsing the text file with NE mark-up
\$RASP-EVAL/parc700/test.ne instead of 
\$RASP-EVAL/parc700/test.not-ne.

\item {\bf -x : translate switch}

This switch specifies that test-file.out file
contains GRs that specify the original word already
(in the case that the user has deliberately parsed
\$RASP-EVAL/parc700/test.(not-)ne.stag file instead
of \$RASP-EVAL/parc700/test.(not-)ne.stag.data 
using the rasp\_parse.sh script. Hence this switch
specifies that the eval\_transform.sh script should
not be applied.

\item {\bf -d : debug switch}

This switch should be used if the user wishes to keep copies
of the x.grtext and x.parses files. Otherwise these files
are removed after the evaluation system terminates.

\item {\bf -e $<$eval\_system parameters$>$}

The -e option specifies (in single quotes, e.g., -e'-o -u') 
the evaluation system parameters to be passed 
to the eval\_system.sh script. These parameters are outlined below.

\end{itemize}

{\bf Eval\_system.sh parameters:}
\begin{itemize}
\item {\bf -t $<$test-file$>$}

This option is used to specify the the test file name (test-file). 
That is, the x in the RASP output file x.out and the input files
x.grtext and x.parses.

\item {\bf -c $<$std-file$>$}

This is the name of the standard file: sents-tst.gr or
sents-ne-tst.gr depending on whether NE markup was used.

\item {\bf -g $<$gr-type match$>$}

The different matching schemes for GR-types is outlined in 
Section~\ref{grmatch} including (e)quality-, (s)ubsumption-
and (h)ierarchy-based matching. 
Hence, the specified parameters
passed with the -g switch are e, s, and h to select
these schemes, respectively. 
The default value is h.

\item {\bf -s $<$slot-compare$>$}

Section~\ref{slot-compare} outlines the various sets of
slots that may be compared including (a)ll, (h)ead-dependent, 
head-dependent-(n)csubj.
Hence, the specified parameters
passed with the -s switch are a, h, and n to select
these schemes, respectively. 
The default value is a.

%\item {\bf -p : pre-process the \stt{ncsubj} GRs}

%This option is available for short-term use until Judita
%can provide code for full post-processing of GRs from RASP.
%If specified, the \stt{ncsubj} and \stt{passive} GRs
%are processed prior to evaluation. Given \stt{(ncsubj Vb x i)}
%and \stt{(passive Vb)} GRs where 
%\stt{Vb} is a given verb and 
%\stt{x} and \stt{i} can hold any value: we remove
%the \stt{(passive Vb)} GR from the GR set and
%ensure that \stt{i} is set to the value \stt{obj}.

\item {\bf -o : output the original evaluation schemes' performance}

This switch enables the user to output the original evaluation
schemes' performance in addition to the new schemes'. Note
that feedback on which GRs matched is provided for each sentence 
as well (i.e. common, missing and extra GR lists outlined
in Section~\ref{eval}).

\item {\bf -u : output the unlabelled dependency performance}

This switch enables the user to output the micro- and macro-averaged
scores for unlabelled dependency performance in addition
to the new schemes'. 
Again, feedback on which GRs matched is provided 
for each sentence as well.

\item {\bf -m : output the confusion matrix to standard-out}

This switch enables the user to output the confusion matrix for the
new scheme specified.

\end{itemize}

\subsubsection{Error Messages}

This section briefly outlines the possible error messages (output
by eval\_system.sh) and why these errors occur:
\begin{itemize}
\item{\bf ``Reached end of file at different times''}

The user has input .grtext .parses and
a standard file in which the number of sentences differs between
files. That is, one or more files does not contain enough sentences
compared to the longest file. The .grtext and/or .parses
files must be modified so that the same number of 
sentences occurs in all three files.

\item{\bf ``Mismatch in sentence numbers: W (text), S (standard), T (test)''}

The user has input .grtext .parses and
a standard file in which the sentences numbers differ at a 
point in the files. The files are expected to contain sentence
numbers starting from 1 and incrementing by 1 until the end of file.
If a sentence number is missing in one of the input files
then this error will occur. The user is provided with the sentence
number specified in the text, standard and test files: W, S, and T,
respectively. The .grtext and/or .parses files must be modified so
that the same sentence numbers occur in order in all three files.

\item{\bf ``Unexpected end of file, stream S''}

The stream S may be one of text-str, std-str, or tst-str
for the .grtext .parses and standard file, respectively.
The stream specified ended while a GR was being read in.
The user should check the respective file's end to correct
the source of error.

%\item{"Expecting number at byte B, stream S"}
%If weighted GRs are input to the program

\item{\bf ``Expecting left parenthesis at byte B, stream S''}

The stream S may be one of text-str, std-str, or tst-str
for the .grtext .parses and standard file, respectively.
The system was expecting to read in `(' (the beginning of
a GR) at byte B but found a different character.
The user should check the respective file's end to correct
the source of error.

\item{\bf ``Unknown relation name G''}

The relation name G is not in the set of possible GR-types allowed
(see Figure~\ref{gr-slots} containing the list of allowable GR-types).
This could occur in any of the input files.

\item{\bf ``Not enough arguments to relation in G - expecting S''}

An input file contains a GR G that does not contain the specified
number of slots. The slots expected are reported as S.

\item{\bf ``Wrong number of slots in relation G - expecting S''}

This error message specifies the same error as the previous
message.
\end{itemize}

\section{Current System Performance}
\label{perf}

This section briefly outlines the `current' performance\footnote{This
performance was measured in December 2005 since which time the system
has been modified. Users can ascertain the performance 
of the system they currently utilise via the scripts outlined herein.}
of the RASP system (without NE markup) measured
using the range of evaluation schemes available.
The performance can be summarised in 
Table~\ref{perftable}, while 
system performance using NE mark-up is shown in 
Table~\ref{NEperftable}.\footnote{Performance for the 
schemes was derived by pre-processing the input. 
Given \stt{(ncsubj Vb x i)} and \stt{(passive Vb)} GRs where 
\stt{Vb} is a given verb and \stt{x} and 
\stt{i} can hold any value: we remove
the \stt{(passive Vb)} GR from the GR set and
ensure that \stt{i} is set to the value \stt{obj}.}

\begin{table*}
\begin{tabular}{|l||c|c||c|c||c|c||}
%\begin{longtable}{|c|c|p{60mm}|p{100mm}|}
\hline
 & \multicolumn{2}{c||}{All} &  \multicolumn{2}{c||}{Head-Dependent-NCSUBJ} &
 \multicolumn{2}{c||}{Head-Dependent}\\
\cline{2-7}
GR-type/Model & Micro-F$_{1}$ & Macro-F$_{1}$ & Micro-F$_{1}$ &
 Macro-F$_{1}$ & Micro-F$_{1}$ & Macro-F$_{1}$ \\
\hline
Original & 71.73 & 61.25 & - & - & - & -\\
\hline
Equality & 71.24 & 56.32 & 71.49  & 56.52 & 71.49  & 56.52 \\
\hline
Subsumption & 71.53 & 59.60 & 71.78 & 59.79 & 71.78 & 59.79 \\
\hline
Hierarchy & 73.68 & 61.98 & 73.94 & 62.21 & 73.94 & 62.21 \\
\hline
Unlabelled & 75.91 & 75.91 &  76.16 &  76.16 & 76.16 &  76.16 \\
\hline
\end{tabular}
\caption{\label{perftable}Parser Performance over Parc DepBank using
different Slot selection schemes and GR-type match schemes.}
%\end{longtable}
\end{table*}

 \begin{table*}
\begin{tabular}{|l||c|c||c|c||c|c||}
%\begin{longtable}{|c|c|p{60mm}|p{100mm}|}
\hline
 & \multicolumn{2}{c||}{All} &  \multicolumn{2}{c||}{Head-Dependent-NCSUBJ} &
 \multicolumn{2}{c||}{Head-Dependent}\\
\cline{2-7}
GR-type/Model & Micro-F$_{1}$ & Macro-F$_{1}$ & Micro-F$_{1}$ &
 Macro-F$_{1}$ & Micro-F$_{1}$ & Macro-F$_{1}$ \\
\hline
Original & 72.96 & 61.65 & - & - & - & -\\
\hline
Equality & 72.47 & 56.75 & 72.74 & 56.96 &  72.74 & 56.96 \\
\hline
Subsumption &  72.78 & 59.97 & 73.05 & 60.19 &  73.05 & 60.19 \\
\hline
Hierarchy &  74.84 &  62.42 & 75.11 & 62.66 & 75.11 & 62.66 \\
\hline
Unlabelled & 77.56 & 77.56 & 77.83 & 77.83 & 77.83 & 77.83  \\
\hline
\end{tabular}
\caption{\label{NEperftable}Parser Performance over Parc DepBank NE
  data using different Slot selection schemes and GR-type match schemes.}
%\end{longtable}
\end{table*}

%current system performance using each evaluation scheme 
%table for each + table comparing F1 for each type.


\bibliography{all-papers-main}

\end{document}


\section{Evaluation Schemes}

\subsection{Original}

------------
Summary

Relation                Precision    Recall    F-score   tst GRs    std GRs
 dependent                71.12      72.44      71.78   10575.00   10382.00
   mod                    62.67      65.66      64.13    3603.00    3439.00
    ncmod                 63.81      68.52      66.08    3308.00    3081.00
    xmod                  50.88      48.88      49.86     171.00     178.00
    cmod                  50.45      33.33      40.14     111.00     168.00
    pmod                  30.77      33.33      32.00      13.00      12.00
  det                     87.54      90.87      89.17    1148.00    1106.00
  arg_mod                 69.04      69.83      69.44    7895.00    7806.00
                          31.25      -5.00     -11.90      16.00       0.00
   arg                    74.56      73.00      73.77    4276.00    4367.00
                           0.00       0.00       0.00       3.00       0.00
    subj                  78.04      66.52      71.82    1161.00    1362.00
     ncsubj               78.73      66.74      72.24    1147.00    1353.00
     xsubj                40.00      28.57      33.33       5.00       7.00
     csubj                11.11      50.00      18.18       9.00       2.00
    subj_or_dobj          80.47      73.26      76.69    2836.00    3115.00
    comp                  73.33      75.94      74.61    3112.00    3005.00
     obj                  77.17      77.00      77.09    2304.00    2309.00
                          54.55     -24.00     -85.71      44.00       0.00
      dobj                82.15      78.49      80.28    1675.00    1753.00
      obj2                15.79      30.00      20.69      38.00      20.00
      iobj                68.01      69.40      68.70     547.00     536.00
     clausal              62.29      72.77      67.12     785.00     672.00
      xcomp               74.55      76.90      75.71     393.00     381.00
      ccomp               50.00      67.35      57.39     392.00     291.00
     pcomp                65.22      62.50      63.83      23.00      24.00
  aux                     92.21      88.75      90.45     385.00     400.00
  conj                    65.24      71.87      68.39     607.00     551.00
  ta                      41.62      52.92      46.60     370.00     291.00
  passive                 94.12      70.18      80.40     170.00     228.00

Micro-averaged
Precision        71.12   Recall  72.44   F-score  71.78   tst GRs  18.88 std GRs  18.54

Macro-averaged
Precision        63.78   Recall  60.48   F-score  62.08

\subsection{Equality Based}

Relation                Precision    Recall    F-score   tst GRs    std GRs
 dependent                70.85      72.16      71.50   10575.00   10382.00
   mod                    62.67      65.66      64.13    3603.00    3439.00
    ncmod                 63.81      68.52      66.08    3308.00    3081.00
    xmod                  50.88      48.88      49.86     171.00     178.00
    cmod                  50.45      33.33      40.14     111.00     168.00
    pmod                  30.77      33.33      32.00      13.00      12.00
  det                     87.54      90.87      89.17    1148.00    1106.00
  arg_mod                 68.68      69.46      69.07    7895.00    7806.00
                           0.00       0.00       0.00      16.00       0.00
   arg                    73.99      72.45      73.22    4276.00    4367.00
                           0.00       0.00       0.00       3.00       0.00
    subj                  78.04      66.52      71.82    1161.00    1362.00
     ncsubj               78.73      66.74      72.24    1147.00    1353.00
     xsubj                40.00      28.57      33.33       5.00       7.00
     csubj                11.11      50.00      18.18       9.00       2.00
    subj_or_dobj          80.47      73.26      76.69    2836.00    3115.00
    comp                  72.56      75.14      73.83    3112.00    3005.00
     obj                  76.13      75.96      76.05    2304.00    2309.00
                           0.00       0.00       0.00      44.00       0.00
      dobj                82.15      78.49      80.28    1675.00    1753.00
      obj2                15.79      30.00      20.69      38.00      20.00
      iobj                68.01      69.40      68.70     547.00     536.00
     clausal              62.29      72.77      67.12     785.00     672.00
      xcomp               74.55      76.90      75.71     393.00     381.00
      ccomp               50.00      67.35      57.39     392.00     291.00
     pcomp                65.22      62.50      63.83      23.00      24.00
  aux                     92.21      88.75      90.45     385.00     400.00
  conj                    65.24      71.87      68.39     607.00     551.00
  ta                      41.62      52.92      46.60     370.00     291.00
  passive                 94.12      70.18      80.40     170.00     228.00

Micro-averaged
Precision        70.85   Recall  72.16   F-score  71.50   tst GRs  18.88 std GRs  18.54

Macro-averaged
Precision        59.01   Recall  60.48   F-score  59.74

with head/dep only

------------
Summary

Relation                Precision    Recall    F-score   tst GRs    std GRs
 dependent                71.12      72.44      71.78   10575.00   10382.00
   mod                    62.67      65.66      64.13    3603.00    3439.00
    ncmod                 63.81      68.52      66.08    3308.00    3081.00
    xmod                  50.88      48.88      49.86     171.00     178.00
    cmod                  50.45      33.33      40.14     111.00     168.00
    pmod                  30.77      33.33      32.00      13.00      12.00
  det                     87.54      90.87      89.17    1148.00    1106.00
  arg_mod                 68.97      69.75      69.36    7895.00    7806.00
                           0.00       0.00       0.00      16.00       0.00
   arg                    74.53      72.98      73.75    4276.00    4367.00
                           0.00       0.00       0.00       3.00       0.00
    subj                  80.02      68.21      73.64    1161.00    1362.00
     ncsubj               80.73      68.44      74.08    1147.00    1353.00
     xsubj                40.00      28.57      33.33       5.00       7.00
     csubj                11.11      50.00      18.18       9.00       2.00
    subj_or_dobj          81.28      74.00      77.47    2836.00    3115.00
    comp                  72.56      75.14      73.83    3112.00    3005.00
     obj                  76.13      75.96      76.05    2304.00    2309.00
                           0.00       0.00       0.00      44.00       0.00
      dobj                82.15      78.49      80.28    1675.00    1753.00
      obj2                15.79      30.00      20.69      38.00      20.00
      iobj                68.01      69.40      68.70     547.00     536.00
     clausal              62.29      72.77      67.12     785.00     672.00
      xcomp               74.55      76.90      75.71     393.00     381.00
      ccomp               50.00      67.35      57.39     392.00     291.00
     pcomp                65.22      62.50      63.83      23.00      24.00
  aux                     92.21      88.75      90.45     385.00     400.00
  conj                    65.24      71.87      68.39     607.00     551.00
  ta                      43.24      54.98      48.41     370.00     291.00
  passive                 94.12      70.18      80.40     170.00     228.00

Micro-averaged
Precision        71.12   Recall  72.44   F-score  71.78   tst GRs  18.88 std GRs  18.54

Macro-averaged
Precision        59.21   Recall  60.69   F-score  59.94

\subsection{Subsumption Based}

------------
Summary

Relation                Precision    Recall    F-score   tst GRs    std GRs
 dependent                71.13      72.45      71.79   10575.00   10382.00
   mod                    62.67      65.66      64.13    3603.00    3439.00
    ncmod                 63.81      68.52      66.08    3308.00    3081.00
    xmod                  50.88      48.88      49.86     171.00     178.00
    cmod                  50.45      33.33      40.14     111.00     168.00
    pmod                  30.77      33.33      32.00      13.00      12.00
  det                     87.54      90.87      89.17    1148.00    1106.00
  arg_mod                 69.06      69.84      69.45    7895.00    7806.00
                          31.25      -5.00     -11.90      16.00       0.00
   arg                    74.58      73.02      73.79    4276.00    4367.00
                          33.33      -1.00      -2.06       3.00       0.00
    subj                  78.04      66.52      71.82    1161.00    1362.00
     ncsubj               78.73      66.74      72.24    1147.00    1353.00
     xsubj                40.00      28.57      33.33       5.00       7.00
     csubj                11.11      50.00      18.18       9.00       2.00
    subj_or_dobj          80.47      73.26      76.69    2836.00    3115.00
    comp                  73.33      75.94      74.61    3112.00    3005.00
     obj                  77.17      77.00      77.09    2304.00    2309.00
                          54.55     -24.00     -85.71      44.00       0.00
      dobj                82.15      78.49      80.28    1675.00    1753.00
      obj2                15.79      30.00      20.69      38.00      20.00
      iobj                68.01      69.40      68.70     547.00     536.00
     clausal              62.29      72.77      67.12     785.00     672.00
      xcomp               74.55      76.90      75.71     393.00     381.00
      ccomp               50.00      67.35      57.39     392.00     291.00
     pcomp                65.22      62.50      63.83      23.00      24.00
  aux                     92.21      88.75      90.45     385.00     400.00
  conj                    65.24      71.87      68.39     607.00     551.00
  ta                      41.62      52.92      46.60     370.00     291.00
  passive                 94.12      70.18      80.40     170.00     228.00

Micro-averaged
Precision        71.13   Recall  72.45   F-score  71.79   tst GRs  18.88 std GRs  18.54

Macro-averaged
Precision        65.63   Recall  60.48   F-score  62.95

with head/dep only:

------------
Summary

Relation                Precision    Recall    F-score   tst GRs    std GRs
 dependent                71.40      72.73      72.06   10575.00   10382.00
   mod                    62.67      65.66      64.13    3603.00    3439.00
    ncmod                 63.81      68.52      66.08    3308.00    3081.00
    xmod                  50.88      48.88      49.86     171.00     178.00
    cmod                  50.45      33.33      40.14     111.00     168.00
    pmod                  30.77      33.33      32.00      13.00      12.00
  det                     87.54      90.87      89.17    1148.00    1106.00
  arg_mod                 69.35      70.14      69.74    7895.00    7806.00
                          31.25      -5.00     -11.90      16.00       0.00
   arg                    75.12      73.55      74.33    4276.00    4367.00
                          33.33      -1.00      -2.06       3.00       0.00
    subj                  80.02      68.21      73.64    1161.00    1362.00
     ncsubj               80.73      68.44      74.08    1147.00    1353.00
     xsubj                40.00      28.57      33.33       5.00       7.00
     csubj                11.11      50.00      18.18       9.00       2.00
    subj_or_dobj          81.28      74.00      77.47    2836.00    3115.00
    comp                  73.33      75.94      74.61    3112.00    3005.00
     obj                  77.17      77.00      77.09    2304.00    2309.00
                          54.55     -24.00     -85.71      44.00       0.00
      dobj                82.15      78.49      80.28    1675.00    1753.00
      obj2                15.79      30.00      20.69      38.00      20.00
      iobj                68.01      69.40      68.70     547.00     536.00
     clausal              62.29      72.77      67.12     785.00     672.00
      xcomp               74.55      76.90      75.71     393.00     381.00
      ccomp               50.00      67.35      57.39     392.00     291.00
     pcomp                65.22      62.50      63.83      23.00      24.00
  aux                     92.21      88.75      90.45     385.00     400.00
  conj                    65.24      71.87      68.39     607.00     551.00
  ta                      43.24      54.98      48.41     370.00     291.00
  passive                 94.12      70.18      80.40     170.00     228.00

Micro-averaged
Precision        71.40   Recall  72.73   F-score  72.06   tst GRs  18.88 std GRs  18.54

Macro-averaged
Precision        65.83   Recall  60.69   F-score  63.15

\subsection{Hiearchy Based}

------------
Summary

Relation                Precision    Recall    F-score   tst GRs    std GRs
 dependent                75.27      76.67      75.97   10575.00   10382.00
   mod                    63.25      66.27      64.73    3603.00    3439.00
    ncmod                 67.02      71.96      69.40    3308.00    3081.00
    xmod                  58.48      56.18      57.31     171.00     178.00
    cmod                  63.06      41.67      50.18     111.00     168.00
    pmod                  61.54      66.67      64.00      13.00      12.00
  det                     89.20      92.59      90.86    1148.00    1106.00
  arg_mod                 73.64      74.48      74.06   15790.00   15612.00
   arg                    76.52      74.93      75.71    4276.00    4367.00
    subj                  77.86      66.37      71.66    1161.00    1362.00
     ncsubj               80.65      68.37      74.00    1147.00    1353.00
     xsubj                60.00      42.86      50.00       5.00       7.00
     csubj                33.33     150.00      54.55       9.00       2.00
    subj_or_dobj          81.06      73.80      77.26    2836.00    3115.00
    comp                  74.87      77.54      76.18    3112.00    3005.00
     obj                  77.99      77.83      77.91    2304.00    2309.00
      dobj                84.36      80.60      82.44    1675.00    1753.00
      obj2                42.11      80.00      55.17      38.00      20.00
      iobj                81.90      83.58      82.73     547.00     536.00
     clausal              62.17      72.62      66.99     785.00     672.00
      xcomp               81.68      84.25      82.95     393.00     381.00
      ccomp               63.27      85.22      72.62     392.00     291.00
     pcomp                82.61      79.17      80.85      23.00      24.00
  aux                     92.47      89.00      90.70     385.00     400.00
  conj                    66.06      72.78      69.26     607.00     551.00
  ta                      50.54      64.26      56.58     740.00     582.00
  passive                 94.12      70.18      80.40     170.00     228.00

Micro-averaged
Precision        74.22   Recall  74.75   F-score  74.48   tst GRs  98.79 std GRs  98.10

Macro-averaged
Precision        70.93   Recall  75.55   F-score  73.17

with head/dependent only:

------------
Summary

Relation                Precision    Recall    F-score   tst GRs    std GRs
 dependent                75.96      77.37      76.66   10575.00   10382.00
   mod                    63.23      66.24      64.70    3603.00    3439.00
    ncmod                 67.38      72.35      69.78    3308.00    3081.00
    xmod                  59.65      57.30      58.45     171.00     178.00
    cmod                  63.96      42.26      50.90     111.00     168.00
    pmod                  61.54      66.67      64.00      13.00      12.00
  det                     89.37      92.77      91.04    1148.00    1106.00
  arg_mod                 73.81      74.65      74.22   15790.00   15612.00
   arg                    76.87      75.27      76.06    4276.00    4367.00
    subj                  79.16      67.47      72.85    1161.00    1362.00
     ncsubj               82.56      69.99      75.76    1147.00    1353.00
     xsubj                60.00      42.86      50.00       5.00       7.00
     csubj                33.33     150.00      54.55       9.00       2.00
    subj_or_dobj          81.59      74.29      77.77    2836.00    3115.00
    comp                  74.87      77.54      76.18    3112.00    3005.00
     obj                  77.99      77.83      77.91    2304.00    2309.00
      dobj                84.60      80.83      82.67    1675.00    1753.00
      obj2                44.74      85.00      58.62      38.00      20.00
      iobj                82.08      83.77      82.92     547.00     536.00
     clausal              62.17      72.62      66.99     785.00     672.00
      xcomp               82.19      84.78      83.46     393.00     381.00
      ccomp               64.54      86.94      74.08     392.00     291.00
     pcomp                78.26      75.00      76.60      23.00      24.00
  aux                     92.47      89.00      90.70     385.00     400.00
  conj                    66.06      72.78      69.26     607.00     551.00
  ta                      54.59      69.42      61.12     740.00     582.00
  passive                 97.06      72.37      82.91     170.00     228.00

Micro-averaged
Precision        74.64   Recall  75.16   F-score  74.90   tst GRs  98.79 std GRs  98.10

Macro-averaged
Precision        71.48   Recall  76.20   F-score  73.77

with ncsubj obj having to match too:

------------
Summary

Relation                Precision    Recall    F-score   tst GRs    std GRs
 dependent                75.85      77.26      76.55   10575.00   10382.00
   mod                    63.23      66.24      64.70    3603.00    3439.00
    ncmod                 67.38      72.35      69.78    3308.00    3081.00
    xmod                  59.65      57.30      58.45     171.00     178.00
    cmod                  63.96      42.26      50.90     111.00     168.00
    pmod                  61.54      66.67      64.00      13.00      12.00
  det                     89.37      92.77      91.04    1148.00    1106.00
  arg_mod                 73.58      74.42      74.00   15790.00   15612.00
   arg                    76.43      74.83      75.62    4276.00    4367.00
    subj                  77.52      66.08      71.34    1161.00    1362.00
     ncsubj               81.52      69.11      74.80    1147.00    1353.00
     xsubj                60.00      42.86      50.00       5.00       7.00
     csubj                33.33     150.00      54.55       9.00       2.00
    subj_or_dobj          80.92      73.68      77.13    2836.00    3115.00
    comp                  74.87      77.54      76.18    3112.00    3005.00
     obj                  77.99      77.83      77.91    2304.00    2309.00
      dobj                84.60      80.83      82.67    1675.00    1753.00
      obj2                44.74      85.00      58.62      38.00      20.00
      iobj                82.08      83.77      82.92     547.00     536.00
     clausal              62.17      72.62      66.99     785.00     672.00
      xcomp               82.19      84.78      83.46     393.00     381.00
      ccomp               64.54      86.94      74.08     392.00     291.00
     pcomp                78.26      75.00      76.60      23.00      24.00
  aux                     92.47      89.00      90.70     385.00     400.00
  conj                    66.06      72.78      69.26     607.00     551.00
  ta                      54.59      69.42      61.12     740.00     582.00
  passive                 97.06      72.37      82.91     170.00     228.00

Micro-averaged
Precision        74.43   Recall  74.95   F-score  74.69   tst GRs  98.79 std GRs  98.10

Macro-averaged
Precision        71.33   Recall  76.06   F-score  73.62

as above + with pre-processing gr lists to get rid of additional passive features

------------
Summary

Relation                Precision    Recall    F-score   tst GRs    std GRs
 dependent                75.10      76.93      76.00   10428.00   10179.00
   mod                    63.20      66.21      64.67    3603.00    3439.00
    ncmod                 66.99      71.92      69.37    3308.00    3081.00
    xmod                  58.48      56.18      57.31     171.00     178.00
    cmod                  63.06      41.67      50.18     111.00     168.00
    pmod                  61.54      66.67      64.00      13.00      12.00
  det                     89.20      92.59      90.86    1148.00    1106.00
  arg_mod                 73.60      74.44      74.02   15790.00   15612.00
   arg                    76.50      74.90      75.69    4276.00    4367.00
    subj                  78.12      66.59      71.90    1161.00    1362.00
     ncsubj               81.17      68.81      74.48    1147.00    1353.00
     xsubj                60.00      42.86      50.00       5.00       7.00
     csubj                22.22     100.00      36.36       9.00       2.00
    subj_or_dobj          81.17      73.90      77.37    2836.00    3115.00
    comp                  74.81      77.47      76.12    3112.00    3005.00
     obj                  77.95      77.78      77.87    2304.00    2309.00
      dobj                84.36      80.60      82.44    1675.00    1753.00
      obj2                42.11      80.00      55.17      38.00      20.00
      iobj                81.90      83.58      82.73     547.00     536.00
     clausal              61.78      72.17      66.58     785.00     672.00
      xcomp               80.92      83.46      82.17     393.00     381.00
      ccomp               63.52      85.57      72.91     392.00     291.00
     pcomp                82.61      79.17      80.85      23.00      24.00
  aux                     93.51      90.00      91.72     385.00     400.00
  conj                    66.06      72.78      69.26     607.00     551.00
  ta                      52.70      67.01      59.00     740.00     582.00
  passive                 78.26      72.00      75.00      23.00      25.00

Micro-averaged
Precision        74.15   Recall  74.83   F-score  74.49   tst GRs  98.27 std GRs  97.37

Macro-averaged
Precision        70.03   Recall  73.90   F-score  71.91
