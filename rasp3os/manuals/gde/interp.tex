\documentstyle[a4,11pt]{report}
%
\newenvironment{display}%
   {\begin{center}
    \begin{minipage}[t]{5.4in}}%
   {\end{minipage}
    \end{center}}
%
\newenvironment{ex}%
   {\begin{display}}%
   {\end{display}}
%
\newcommand{\bs}{$\backslash$}
%
\begin{document} 
\chapter*{ANLT Example Semantic Scoping and Post-Processing Module}

The simple scoping and post-processing module supplied on the Tools
distribution tape (in the file {\it applic/interp.lsp}) is intended to
demonstrate how the output produced by the parser using the ANLT grammar
can be converted into a representation that is easier to deal with and
more suitable for processing by user applications.

To use the module, the source file should first be compiled
(using the Lisp function {\tt compile-file} passing as argument the name
of the file as a string); the resulting object file can then be loaded into
GDE sessions when it is wanted using the Lisp function {\tt load} (executing
this with the GDE `!' command). Once loaded, scoping and post-processing
of the semantic representations (SRs) resulting from successful parses can be
invoked by the GDE parser command `interpret'. E.g.
\begin{display}
{\small\tt
Gde> {\bf !(load "applic/interp")}\\
\\
Gde> {\bf parse}\\
\\
Parse>> she helps\\
100 msec CPU, 110 msec elapsed\\
106 edges generated\\
1 parse\\
\\
(((she) ((help +s))))\\
\\
Parse>> {\bf view semantics}
\begin{verbatim}
(DECL
   (HELP (uqe (some (e1) (PRES e1)))
      (pro (the (x1) (and (sg x1) (female x1))))))
\end{verbatim}
Parse>> {\bf interpret}
\begin{verbatim}
(DECL
   (the (x1) (and (sg x1) (female x1))
      (some (e1) (PRES e1) (HELP e1 x1))))

\end{verbatim}
} \end{display}
The module consists of two parts. The first is a simple quantifier scoper, and
the second, applying to the output of this, is a post-processor which
carries out a set of simplifications.
 

\section*{Scoping}

The scoper produces a single scoping of an SR, where the order of scoped
quantifiers simply reflects their left to right order in the input. The only
constraint is that a quantified event (marked with {\tt uqe})
associated with a proposition or embedded proposition is assigned
narrower scope than all the other quantifiers associated with that
proposition.

Scoping is local to an embedded proposition in that the quantifier which
is the translation of an embedded clause is not scoped outside of the
predicate of the higher clause. For example, the SR assigned by the grammar
to the sentence ``an abbot knows an acquaintance helps" is:
\begin{ex}
\begin{verbatim}
(DECL 
   (KNOW (uqe (some (e1) (PRES e1)))
      (uq (some (x1) (and (sg x1) (ABBOT x1)))) 
      (HELP (uqe (some (e2) (PRES e2)))
         (uq (some (x2) (and (sg x2) (ACQUAINTANCE x2)))))))
\end{verbatim}
\end{ex}
which receives the scoping:
\begin{ex}
\begin{verbatim}
(DECL 
   (some (x1) (and (sg x1) (ABBOT x1)) 
      (some (e1) (PRES e1) 
         (KNOW e1 x1 
            (some (x2) (and (sg x2) (ACQUAINTANCE x2)) 
               (some (e2) (PRES e2) (HELP e2 x2)))))))
\end{verbatim}
\end{ex}
Representations of noun phrases deriving from quantifiers other than {\tt the},
{\tt some} or {\tt every} have a definiteness specification {\tt def} inserted
when appropriate. The default is assumed to be indefinite. Likewise, 
represenations of kind interpretations have a {\tt generic} specification
added.


\section*{Post-Processing}

The post-processor takes a scoped SR and performs a set of simplifications
on it. The simplifications applied are:

\begin{itemize}
\item A formula representing a named person is replaced by just the name
itself, e.g. {\tt (name (the (x1) (and (sg x1) (named x1 KIM) (animate x1))))}
becomes {\tt KIM}.

\item A formula representing a number is replaced by the number itself, e.g.
the representation of the string ``three hundred and sixty five",
{\tt (NN (+ (* \bs 3 \bs 100) (+ \bs 60 \bs 5)))}, becomes {\tt 365}. A
number formula can also be a range, e.g. {\tt (NN (RANGE (NN \bs 2)
(NN \bs 3)))}, which becomes {\tt (RANGE 2 3)}.

\item A non-atomic formula representing a complex predicate (typically
a phrasal verb) is replaced by an appropriate atomic predicate name, e.g.
{\tt (CP BRING BACK)} becomes {\tt BRING-BACK}.

\item A non-atomic formula representing a compound is replaced by
an appropriate atomic name, e.g. {\tt (compound BUTTER MOUNTAIN)} becomes
{\tt BUTTER\_MOUNTAIN}.

\item A formula that is recognised as representing a possession relation is
replaced by a {\tt POSSESSED-BY} predicate, e.g. {\tt (lambda (R1) (R1 x1 KIM))}
becomes {\tt (POSSESSED-BY x1 KIM)}.

\item A formula produced from the scoping of a complex quantifer corresponding
to a comparative or adjective of degree, such as ``more abbots than ..."
\begin{ex}
\begin{verbatim}
(((more than)
    (lambda (c1)
       (and (pl c1) (ABBOT c1) (some (e2) ...))))
 (x2) (and (pl x2) (ABBOT x2)) ...)
\end{verbatim}
\end{ex}
is replaced by
\begin{ex}
\begin{verbatim}
(((more than)
    (?quant (c1) (and (pl c1) (ABBOT c1)) (some (e2) ...)))
 (x2) (and (pl x2) (ABBOT x2)) ...)
\end{verbatim}
\end{ex}

\item Nested {\tt and}s are simplified.
\end{itemize}

\end{document}
